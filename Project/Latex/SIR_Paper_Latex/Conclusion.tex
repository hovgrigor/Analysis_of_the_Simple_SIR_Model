


\section{Conclusion}\label{Conclusion}\thispagestyle{SectionFirstPage}

\hspace{\parindent} In conclusion, after thorough research and analysis of the SIR model, we were able to apply it to predict real-world data.
Even though we demonstrated this with the simple SIR model, the same idea can be extended to more complex models that can predict epidemics more accurately.
There are also better results available that use AI to make the model even more accurate.
\par Through solving the SIR system of ordinary differential equations analytically and then, numerically, we were able to compare the possible outcome given the
parameters and initial values, and check if the SIR model provides a dependable source to rely on for epidemic clinical research.
According to the graphs we obtained by comparing Euler's method, Backwards Euler's method, and the Analytical solution given the necessary initial data, we obtained almost identical results and applied one of the methods to the real data.
to check whether the error between the real COVID-19 data and our prediction is acceptable.
\par The tests came in as a success, and even though there are still errors present between the approximation and the real data, we can confidently say that using the model
with one of the given methods in the early stage of epidemics will provide us with insights into how big of a threat
the disease is and how concerned the researchers should be, hence getting a more precise insight on
the regulations of the precautionary behavior for the whole population.
Our analysis also proves that the SIR model and its extensions are relevant in informing public health policies and interventions.
\par This analysis serves as proof of concept, meaning that more research into the SIR model will allow for better results.
Even with the simple SIR model and using half the data from the COVID-19 epidemic, we were partially successful in predicting the future behavior of the disease.