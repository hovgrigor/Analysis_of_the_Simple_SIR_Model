\section{Introduction}\label{Introduction}\thispagestyle{SectionFirstPage} % Hide headers on the first page of the section
\lhead{Introduction to Epidemic Modeling}
\hspace{\parindent} Infectious disease epidemics cause significant challenges to public health, necessitating mathematical models for understanding and managing their spread.
The practical use of epidemic models must rely heavily on the realism put into the models.
This means that a reasonable model can include only some possible effects but rather incorporate the mechanisms in the most straightforward possible fashion to maintain significant components that influence disease propagation.
Before epidemic models are used to predict real phenomena, great care should be taken.
However, even simple models should, and often do, pose important questions about the underlying mechanisms of infection spread and possible means of control of the disease or epidemic.
There are classical papers by W. Kermack and A. McKendrick \cite{Kermack} that have greatly influenced the development of mathematical models for disease spread and are still in many epidemic situations.
These first papers laid a foundation for modeling infections that confer complete immunity after recovery (or, in case of lethal diseases - death).
The population is taken to be constant - no births or deaths other than from the disease are possible - consistent with the course of an epidemic being short compared with an individual's lifetime.
Suppose a group of infected individuals is introduced into a large population. In that case, a fundamental problem is to describe the spread of the infection within the population as a function of time.
Over time, the epidemic may come to an end.
One of the most critical questions in epidemiology is to ascertain whether this occurs only when all the initially susceptible individuals have contracted the disease or if some interplay of infectivity, recovery, and mortality factors may result in an epidemic “die out” with many susceptibles still present in the unaffected population.
In their first paper, Kermack and McKendrick start with the assumption that all members of the community are initially equally susceptible to the disease and that complete immunity is conferred after the infection.
The population is divided into three distinct classes: the susceptibles, \textit{S}, - healthy individuals who can catch the disease; the infected, \textit{I}, - those who have the disease and can transmit it; and the removed, \textit{R}, - individuals who have had the disease and are now immune to the infection (or removed from further propagation of the disease by some other means).
Schematically, the individual goes through consecutive states \textit{S → I → R}.
Such models are often called the \textit{SIR} models.
