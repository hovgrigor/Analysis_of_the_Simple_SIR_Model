\section{Numerical Methods for Solving the SIR Model}\label{Numerical_Methods}\thispagestyle{SectionFirstPage} % Hide headers on the first page of the section
\subsection{Yet another subsection}\label{multicollinearity}
\setlength{\parindent}{20pt}
\subsection{Determining the constants}
\hspace{\parindent}In order to make the SIR model work, we need to have methods of approximating the parameters $\beta$ - the probability
per unit of time that an infectious quarter will infect a noninfected quarter, and $\gamma$ - the probability
per unit of time that an infectious quarter will recover from the disease.
\par
\par To calculate the $\gamma$, we start by using the formula \ref{eq:SIR}, and supposing that I is a constant
\begin{equation}
	\frac{\partial r}{\partial t} = \gamma I_0 \label{eq:4.2.1}
\end{equation}
\par By integrating \ref{eq:4.2.1} we will obtain
\begin{equation}
	r(t) = \gamma t I_0 \label{eq:4.2.2}
\end{equation}
\par At t = T days, such as $R(T)=I_0$, we obtain
\begin{equation}
	r = \gamma T I_0 \label{eq:4.2.3}
\end{equation}

\par By getting the derivative of r with respect of t, we get

\begin{equation}
	 \frac{\partial r}{\partial t} = \gamma I(t) = \frac{r(t+\delta t)-r(t))}{\delta t}\label{eq:4.2.4}
\end{equation}
\par We pick $\delta$t as 1, and get the formula for $\gamma$
\begin{equation}
	 \gamma \approx \frac{r(t+1)-r(t)}{I(t)}
\end{equation}
\par In order to determine the $\beta$, we are going to use the formula \ref{eq:SIR} and consider the
time t, when the disease enters a population and the number of susceptibles is equal to the population itself, $S\cong N$
\begin{equation}
	I^{\prime} = (\beta - \gamma) I \label
\end{equation}
\par let $k=\beta-\gamma$, hence $I^{\prime} \cong kI$. After solving this ODE, we get the form
\begin{equation}
	I(t)=I_{0}e^{kt}
\end{equation}
\par from here
\begin{equation}
	ln(I(t))=ln(I_{0}) + kt
\end{equation}
\par hence k can be determined by the equation
\begin{equation}
	k = \frac{ln(I(t))-ln(I_{0})}{t}
\end{equation}
\par So, as we have that $k=\beta-\gamma$, we can find $\beta$ by;
\begin{equation}
	\beta = k - \gamma
\end{equation}
\par where $\gamma$ and $k$ can be determined from a real sample of data. \cite{Math_Hands-On_with_Python}