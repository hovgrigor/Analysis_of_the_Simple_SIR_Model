\section{Numerical Methods for Solving the SIR Model}\label{Numerical_Methods}\thispagestyle{SectionFirstPage} % Hide headers on the first page of the section
\subsection{Solving Numerically Using Euler's Method}\label{multicollinearity}
\hspace{\parindent} One of the most known ways of solving an ODE system, is Euler's method. This method is solving the IVT;
\begin{equation}
	\left\{\begin{aligned}
		& y'(x) = f(x,y(x)), x \in [a,b] \\
		& y'(x_0)=y_0 \\
	\end{aligned}
\end{equation}

\par here we are taking a mesh of $x$ in the necessary range and calculating the $y(x)$ only at points of $x_K$
After that, using the initial $y_0$, we approximate the next $y$, by using the formula
\begin{equation}
	y_{k+1} = y_k + h f(x_k,y_k)
\end{equation}
\par where $h=\frac{b-a}{n}$ is the stepsize of the $x$.
\par In our case, we are going to solve 3 such IVT's;
\[
	\left\{
		\begin{array}{l}
			\frac{\partial s}{\partial t} = -\beta s(t)  i(t)  \\
			s(t_0)=s_0
		\end{array}
	\right.
\]

\[
	\left\{
		\begin{array}{l}
			\frac{\partial i}{\partial t} = \beta s(t)  i(t) - \gamma i(t) \\
			i(t_0)=i_0
		\end{array}
	\right.
\]

\[
	\left\{
		\begin{array}{l}
			\frac{\partial r}{\partial t} = \gamma i(t) \\
			r(t_0)=r_0
		\end{array}
	\right.
\]

\par As you can see, we are going to need the initial $t_0$ value, as well as the number
of susceptible, infected, and recovered population at that $t_0$.
After obtaining the sufficient pieces of information, we can use the Euler's formula and obtain

\[
	\left\{
		\begin{array}{l}
			s_{k+1}=s_k - \beta \cdot h \cdot s_k \cdot i_k \\
			i_{k+1}=i_k + (\beta \cdot s_k \cdot i_k - \gamma \cdot i_k)h \\
			r_{k+1}=r_k+ \gamma \cdot h \cdot i_k
		\end{array}
	\right.
\]
\par
\par Thus we are going to be able to determine the needed values of $s(t)$, $i(t)$ and $r(t)$
at the necessary $t_k$ date by doing the necessary amount of iterations.

\subsection{Backward Euler's Method}
\par Just like forward Euler's Method described previously, backward Euler's method also aims to solve
the IVT

\begin{equation}
	\left\{\begin{aligned}
		& y'(x) = f(x,y(x)), x \in [a,b] \\
		& y'(x_0)=y_0 \\
	\end{aligned}
\end{equation}

\par But this time, y


\subsection{Determining the constants}
\hspace{\parindent}In order to make the SIR model work, we need to have methods of approximating the parameters $\beta$ - the probability
per unit of time that an infectious quarter will infect a noninfected quarter, and $\gamma$ - the probability
per unit of time that an infectious quarter will recover from the disease.
\par
\par To calculate the $\gamma$, we start by using the formula \ref{eq:SIR}, and supposing that I is a constant
\begin{equation}
	\frac{\partial r}{\partial t} = \gamma I_0 \label{eq:4.2.1}
\end{equation}
\par By integrating \ref{eq:4.2.1} we will obtain
\begin{equation}
	r(t) = \gamma t I_0 \label{eq:4.2.2}
\end{equation}
\par At t = T days, such as $R(T)=I_0$, we obtain
\begin{equation}
	r = \gamma T I_0 \label{eq:4.2.3}
\end{equation}

\par By getting the derivative of r with respect of t, we get

\begin{equation}
	 \frac{\partial r}{\partial t} = \gamma I(t) = \frac{r(t+\delta t)-r(t))}{\delta t}\label{eq:4.2.4}
\end{equation}
\par We pick $\delta$t as 1, and get the formula for $\gamma$
\begin{equation}
	 \gamma \approx \frac{r(t+1)-r(t)}{I(t)}
\end{equation}
\par In order to determine the $\beta$, we are going to use the formula \ref{eq:SIR} and consider the
time t, when the disease enters a population and the number of susceptibles is equal to the population itself, $S\cong N$
\begin{equation}
	I^{\prime} = (\beta - \gamma) I \label
\end{equation}
\par let $k=\beta-\gamma$, hence $I^{\prime} \cong kI$. After solving this ODE, we get the form
\begin{equation}
	I(t)=I_{0}e^{kt}
\end{equation}
\par from here
\begin{equation}
	ln(I(t))=ln(I_{0}) + kt
\end{equation}
\par hence k can be determined by the equation
\begin{equation}
	k = \frac{ln(I(t))-ln(I_{0})}{t}
\end{equation}
\par So, as we have that $k=\beta-\gamma$, we can find $\beta$ by;
\begin{equation}
	\beta = k - \gamma
\end{equation}
\par where $\gamma$ and $k$ can be determined from a real sample of data. \cite{Math_Hands-On_with_Python}