\section{Conclusion}\label{Conclusion}\thispagestyle{SectionFirstPage} % Hide headers on the first page of the section
\hspace{\parindent} In conclusion, our exploration of the SIR model, both
analytically and numerically, has provided valuable insights
into its dynamics and applications, particularly in the context
of understanding and combating pandemics at the early stage and predicting the spread
 and the numbers of infected, susceptible, and recovered quarters over time, hence
being ready to make the necessary decision of precautionary processes.
\par Through solving the SIR system of ordinary differential equations analytically, and
then numerically, we were able to compare the possible outcome given the parameters and initial values,
 and check if the SIR model gives a dependable source to rely on the pandemic clinical research.
According to the graphs we obtained through comparing the Euler's method, backward Euler's method, and the Analytical solution
given the necessary initial data, we obtained almost identical results and decided to apply one of the methods to the real data
to see if it is possible to get a small error with the real data on COVID-19.
\par The tests came in successful, and even though there are still errors present between the
approximation and the real data, we can confidently say, that using the model with
one of the given methods in the early stage of pandemics will give us insights into how big of a threat
the disease is and how concerned should the researchers be, hence getting a more precise insight on
the regulations of the precautionary behavior for the whole population.
\par Thus the practical relevance of the SIR model in informing public health policies
and interventions were proved as well. By accurately capturing the dynamics of disease transmission
and the impact of control measures, the SIR model can aid policymakers in making
informed decisions to mitigate the spread of infectious diseases and minimize their societal impact.
\par Looking ahead, our study underscores the importance of continued research into mathematical
models like the SIR model, not only for understanding current pandemics like COVID-19 but also
for preparedness against future infectious disease outbreaks. By refining and extending these
models, incorporating additional complexities and real-world data, we can enhance our ability
to forecast epidemic trends, evaluate intervention strategies, and ultimately safeguard public
health on a global scale.