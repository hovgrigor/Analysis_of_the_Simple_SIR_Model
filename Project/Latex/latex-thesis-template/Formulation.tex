\section{Mathematical Formulation of the SIR Model}\label{Mathematical_Formulation}\thispagestyle{SectionFirstPage} % Hide headers on the first page of the section
\lhead{Mathematical Formulation of the SIR Model}
\setlength{\parindent}{20pt}
\subsection{Types of Variables}
\hspace{\parindent}The first step in mathematical modeling process is defining independent and dependent variables.
The independent variable is time $t$, measured in days.
Afterward, two related sets of dependent variables are being defined.
The first set of dependent variables counts people in each of the groups, each as a function of time:\\
\begin{center}
    $S = S(t)$ is the number of \textit{susceptible} individuals,\\
    $I = I(t)$ is the number of \textit{infected} individuals, and\\
    $R = R(t)$ is the number of \textit{recovered} individuals.
\end{center}
The second set of dependent variables represents the \textit{fraction} of the total population in each of the three categories.
So, if $N$ is the total population we have:\\
\begin{center}
    $s(t) = \frac{S(t)}{N}$ the susceptible fraction of the population,\\
    $i(t) = \frac{I(t)}{N}$ the infected fraction of the population, and\\
    $r(t) = \frac{R(t)}{N}$ the recovered fraction of the population.
\end{center}
Using fractions instead of population counts simplifies calculations.
As the two sets of dependent variables are proportional to each other, using either set will give us the same information about the progress of the epidemic.\\

\subsection{Important Assumptions}
\hspace{\parindent}The next step is making some important assumptions on the rates of changes of our dependent variables.\\
Natural deaths, births, immigration and other similar factors are being ignored, hence no one is being \textit{added} to the susceptible group.\\
The only way an individual \textit{leaves} the susceptible group is by becoming infected.\\
The rate of change of the number of susceptible individuals, $S(t)$, over time depends on the existing number of susceptibles, the number of individuals currently infected, and the level of interaction between susceptibles and infected individuals.
Specifically, we assume that each infected individual initiates a fixed number $\beta$ of contacts per day that can potentially transmit the disease.
However, not all of these contacts are made with susceptible individuals.
Assuming a uniformly mixed population, the proportion of these contacts involving susceptibles is denoted by $s(t)$.
Therefore, on average, each infected individual gives rise to $\beta \cdot s(t)$ new daily infections.
This formulation simplifies complex scenarios, such as instances where a single susceptible individual encounters multiple infected individuals within a single day, particularly where the susceptible population dramatically outnumbers the infected population.\\
\\
A fixed fraction $\gamma$ of the infected group will be recovered during any given day.
For instance, assuming an average infection duration of three days, approximately one-third of the presently infected population typically transitions to the recovered state daily.
It is important to clarify that when referring to ``infected,'' the term specifically denotes ``infectious,'' indicating individuals capable of transmitting the disease to susceptible individuals.
Conversely, individuals classified as ``recovered'' may still experience discomfort, and there remains a risk of complications such as pneumonia-related fatalities at a later stage.
These assumptions inform the derivatives of our dependent variables.
\begin{equation}
    \frac{\partial s}{\partial t} = -\beta s(t)  i(t)
\end{equation}
\begin{equation}
    \frac{\partial r}{\partial t} = \gamma i(t)
\end{equation}

As the sum of \textit{susceptable, infected and recovered} people gives the whole population, it means that\\
\begin{equation}
    \frac{\partial s}{\partial t} + \frac{\partial i}{\partial t}+\frac{\partial r}{\partial t} = 0
\end{equation}
To get the differential equation for \textit{infecteds} it is enough to plug equations $(1)$ and $(2)$ into equation $(3)$.
The result will be:
\begin{equation}
    \frac{\partial i}{\partial t} = \beta s(t)  i(t) - \gamma i(t)
\end{equation}
Combining all three equations will give the following system of differential equations:
\[
\left\{
    \begin{array}{l}
        \frac{\partial s}{\partial t} = -\beta s(t)  i(t)\\
        \frac{\partial i}{\partial t} = \beta s(t)  i(t) - \gamma i(t)\\
        \frac{\partial r}{\partial t} = \gamma i(t)
    \end{array}
\right.
\]
\subsection{Extensions and Modifications}
\hspace{\parindent}The basic SIR model can be expanded and modified in various ways to incorporate factors such as death and birth rates, vaccination rates, reinfection rates, and other significant variables, leading to more accurate calculations and predictions.
One famous modification was made by Yamamoto in 2018.
Let death rate be denoted by $\alpha$ and death rate be denoted by $\mu$.
The birth rate contributes to an increase in the size of the susceptible population, while the death rate leads to a decrease in its size.
The extent of this change can be calculated as:
\begin{equation*}
    \frac{\partial s}{\partial t} = \alpha - \mu s(t) - \beta s(t)i(t)
\end{equation*}
\hspace{\parindent}Yamamoto made another modification of the SIR model.
In this modification he took into consideration the inoculation rate of vaccine.
Let $\varepsilon$ be the rate at which individuals are vaccinated within a given population.
Incorporating this parameter leads to an increase in the population within the recovered group, thus resulting in the following modifications to the model equations:
\begin{equation*}
    \frac{\partial s}{\partial t} = \alpha(1 - \varepsilon) - \mu s(t)i(t)
\end{equation*}
\begin{equation*}
    \frac{\partial i}{\partial t} = \beta s(t)i(t) - (\gamma + \mu)(t)
\end{equation*}
\begin{equation*}
    \frac{\partial r}{\partial t} = \alpha\varepsilon + \gamma i(t) - \mu r(t)
\end{equation*}
\hspace{\parindent}Satoru Yamamoto examined the combined impact of both vaccination and the rate of reinfection as well.
A small portion of the population might get reinfected at a reinfaction rate $\sigma$.
The equation will look in the following way after making the modification:
\begin{equation*}
    \frac{\partial s}{\partial t} = \alpha(1 - \varepsilon) - \mu s(t)i(t)
\end{equation*}
\begin{equation*}
    \frac{\partial i}{\partial t} = \beta[s(t) + \sigma r(t)]i(t) - (\gamma + \mu)(t)
\end{equation*}
\begin{equation*}
    \frac{\partial r}{\partial t} = \alpha \varepsilon + \gamma i(t) - \mu r(t) - \beta \sigma r(t)i(t)
\end{equation*}
\hspace{\parindent}There is an extension of the SIR model which gives more realistic result.
The model is called SEIR, where ``E'' stands for ``exposed''.
The intensity of the virus depends on the exposure of an individual to the symptomatic asymptomatic carrier of the virus.
According to Peter Turchin there is a need to add another group of exposed individuals to have a more effective model.
That group stage would be between susceptible and infected stages.
It will represent a group of individuals who have encountered the infection but are currently not infected.
Let $e(t)$ be the fraction of the population which is exposed to infection but is not infected.
As it was obtained above, for the standard SIR model the rate of change for the susceptible group is:
\begin{equation*}
    \frac{\partial s}{\partial t} = -\beta s(t)  i(t)
\end{equation*}
Let $\delta$ be the coefficient which shows the possibility of an exposed person contracting the infection.
Then the rate of change of the exposed group can be calculated as:
\begin{equation*}
    \frac{\partial e}{\partial t} = \beta s(t)  i(t) - \delta e(t)
\end{equation*}
Since there is a possibility for exposed individuals to contract the infection, the influence of \textit{susceptibles} on the rate of infected persons is positive.
Additionally, considering the possibility for infected individuals to recover, the impact of \textit{recovery} on the rate of infected persons is negative.
Thus:
\begin{equation*}
    \frac{\partial i}{\partial t} = \delta e(t) - \gamma i(t)
\end{equation*}
Similarly, the rate of change of the recovered group is:
\begin{equation*}
    \frac{\partial r}{\partial t} = \gamma i(t)
\end{equation*}
Combining all the equations above will ge the following system of differential equations:
\[
\left\{
    \begin{array}{l}
        \frac{\partial s}{\partial t} = -\beta s(t)  i(t)\\
        \frac{\partial e}{\partial t} = \beta s(t)  i(t) - \delta e(t)\\
        \frac{\partial i}{\partial t} = \delta e(t) - \gamma i(t)\\
        \frac{\partial r}{\partial t} = \gamma i(t)
    \end{array}
\right.
\]




