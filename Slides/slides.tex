% This text is proprietary.
% It's a part of presentation made by myself.
% It may not used commercial.
% The noncommercial use such as private and study is free
% Dec 2007
% Author: Sascha Frank
% University Freiburg
% www.informatik.uni-freiburg.de/~frank/
%
%
\documentclass{beamer}
\setbeamertemplate{page number in head/foot}[totalframenumber]


\usetheme{Marburg}


\beamersetuncovermixins{\opaqueness<1>{25}}{\opaqueness<2->{15}}
\begin{document}
\title{Mathematical Theory of Infectious Disease Epidemics}
\author{G. Hovhannisyan\\
A. Abrahamyan\\
M. Khachatryan}
\date{\today}


\begin{frame}
\titlepage
\end{frame}

\begin{frame}\frametitle{Table of contents}\tableofcontents
\end{frame}


\section{Introduction}
\begin{frame}\frametitle{Title}
\begin{center}
    Each frame should have a title.
\end{center}
\end{frame}


\section{Mathematical Formulation}
\begin{frame}\frametitle{Mathematical Formulation}
\begin{center}
    \textbf{Mathematical Formulation}
\end{center}
\end{frame}


\begin{frame}\frametitle{Mathematical Formulation}
\begin{itemize}
\item Types of Variables
\item Important Assumptions
\end{itemize}
\end{frame}

\subsection{Types of Variables}
\begin{frame}\frametitle{Independent and Dependent Variables}
    asd
\end{frame}

\subsection{Important Assumptions}
\begin{frame}\frametitle{Importnat Assumptions}
    asd
\end{frame}


\section{Analytical Solution}
\subsection{Tables}
\begin{frame}\frametitle{Tables}
\begin{tabular}{|c|c|c|}
\hline
\textbf{Date} & \textbf{Instructor} & \textbf{Title} \\
\hline
WS 04/05 & Sascha Frank & First steps with  \LaTeX  \\
\hline
SS 05 & Sascha Frank & \LaTeX \ Course serial \\
\hline
\end{tabular}
\end{frame}


\begin{frame}\frametitle{Tables with pause}
\begin{tabular}{c c c}
A & B & C \\
\pause
1 & 2 & 3 \\
\pause
A & B & C \\
\end{tabular}
\end{frame}


\section{Numerical Methods for Solving the SIR}
\subsection{blocs}
\begin{frame}\frametitle{blocs}


\begin{exampleblock}{title of the bloc}
bloc text
\end{exampleblock}


\begin{alertblock}{title of the bloc}
bloc text
\end{alertblock}
\end{frame}


\section{Simulation Results on Real-World Data}
\subsection{split screen}

\begin{frame}\frametitle{splitting screen}
\begin{columns}
\begin{column}{5cm}
\begin{itemize}
\item Beamer
\item Beamer Class
\item Beamer Class Latex
\end{itemize}
\end{column}
\begin{column}{5cm}
\begin{tabular}{|c|c|}
\hline
\textbf{Instructor} & \textbf{Title} \\
\hline
Sascha Frank &  \LaTeX \ Course 1 \\
\hline
Sascha Frank &  Course serial  \\
\hline
\end{tabular}
\end{column}
\end{columns}
\end{frame}

\subsection{Pictures}
\begin{frame}\frametitle{pictures in latex beamer class}
\begin{figure}
\caption{show an example picture}
\end{figure}
\end{frame}

\subsection{joining picture and lists}

\begin{frame}
\frametitle{pictures and lists in beamer class}
\begin{columns}
\begin{column}{5cm}
\begin{itemize}
\item<1-> subject 1
\item<3-> subject 2
\item<5-> subject 3
\end{itemize}
\vspace{3cm}
\end{column}
\begin{column}{5cm}
\begin{overprint}
\end{overprint}
\end{column}
\end{columns}
\end{frame}


\subsection{pictures which need more space}
\begin{frame}[plain]
\frametitle{plain, or a way to get more space}
\begin{figure}
\caption{show an example picture}
\end{figure}
\end{frame}



\end{document}