% This text is proprietary.
% It's a part of presentation made by myself.
% It may not used commercial.
% The noncommercial use such as private and study is free
% Dec 2007
% Author: Sascha Frank
% University Freiburg
% www.informatik.uni-freiburg.de/~frank/
%
%
\documentclass{beamer}
\setbeamertemplate{page number in head/foot}[totalframenumber]


\usetheme{Marburg}


\beamersetuncovermixins{\opaqueness<1>{25}}{\opaqueness<2->{15}}
\begin{document}
\title{Mathematical Theory of Infectious Disease Epidemics}
\author{G. Hovhannisyan & A. Abrahamyan\\
M. Khachatryan}
\date{\today}


\begin{frame}
\titlepage
\end{frame}

\begin{frame}\frametitle{Table of contents}\tableofcontents
\end{frame}


\section{Introduction}
\begin{frame}\frametitle{Title}
\begin{center}
    Each frame should have a title.
\end{center}
\end{frame}


\section{Mathematical Formulation}
\begin{frame}\frametitle{Mathematical Formulation}
\begin{center}
    \textbf{Mathematical Formulation}
\end{center}
\end{frame}


\begin{frame}\frametitle{Mathematical Formulation}
\begin{itemize}
\item Types of Variables
\item Important Assumptions
\end{itemize}
\end{frame}

\subsection{Types of Variables}
\begin{frame}\frametitle{Independent and Dependent Variables}
    The only dependent variable is $t$(time) measured in days.\\
    \vspace{0.15in}
    The first set of dependent variables are functions of time, which count people in each group.
    $$S(t)\text{ is the number of \textit{susceptible} individuals at given time,}$$
    $$I(t)\text{ is the number of \textit{infected} individuals at give time,}$$
    $$R(t)\text{ is the number of \textit{recovered} individuals at give time.}$$
\end{frame}

\begin{frame}\frametitle{Independent and Dependent Variables}
The second set of dependent variables represents the \textit{fraction} of the total population in
each of the three categories. So, if $N$ is the total population we have:
    $$s(t) = \frac{S(t)}{N}\text{ the \textit{susceptible fraction} of the population,}$$
    $$i(t) = \frac{I(t)}{N}\text{the \textit{infected fraction} of the population, and}$$
    $$r(t) = \frac{R(t)}{N}\text{the \textit{recovered fraction} of the population.}$$
\end{frame}

\subsection{Important Assumptions}
\begin{frame}\frametitle{Importnat Assumptions}
    \begin{itemize}
        \item Natural births, deaths, immigration and other similar factors are being ignored.
        \item The only way an individual leaves the susceptible group is by becoming infected.
        \item The rate of change of the number of susceptible individuals, $S(t)$, over time depends on the existing number of susceptibles, the number of individuals currently infected, and the level of interaction between susceptibles and infected individuals.
        \item Each infected individual initiates a fixed number $\beta$ of contacts per day that can potentially transmit the disease.
        \item A uniformly mixed population, the proportion of these contacts involving susceptibles is denoted by $s(t)$. Therefore, on average, each
infected individual gives rise to $\beta \cdot s(t)$ new daily infections.
        \item A fixed fraction $\gamma$ of the infected group will be recovered during any given day.
    \end{itemize}
\end{frame}

\begin{frame}[t]\frametitle{Importnat Assumptions}
These assumptions inform the derivatives of our dependent variables.
    \begin{itemize}
        \item $\frac{\partial{s}}{\partial{t}}=-\beta s(t)i(t)$
        \item $\frac{\partial{r}}{\partial{t}}=\gamma i(t)$
    \end{itemize}
As the sum of \textit{susceptable}, \textit{infected} and \textit{recovered} people gives the whole population, it means that
\end{frame}


\section{Analytical Solution}
\subsection{Tables}
\begin{frame}\frametitle{Tables}
\begin{tabular}{|c|c|c|}
\hline
\textbf{Date} & \textbf{Instructor} & \textbf{Title} \\
\hline
WS 04/05 & Sascha Frank & First steps with  \LaTeX  \\
\hline
SS 05 & Sascha Frank & \LaTeX \ Course serial \\
\hline
\end{tabular}
\end{frame}


\begin{frame}\frametitle{Tables with pause}
\begin{tabular}{c c c}
A & B & C \\
\pause
1 & 2 & 3 \\
\pause
A & B & C \\
\end{tabular}
\end{frame}


\section{Numerical Methods for Solving the SIR}
\subsection{blocs}
\begin{frame}\frametitle{blocs}


\begin{exampleblock}{title of the bloc}
bloc text
\end{exampleblock}


\begin{alertblock}{title of the bloc}
bloc text
\end{alertblock}
\end{frame}


\section{Simulation Results on Real-World Data}
\subsection{split screen}

\begin{frame}\frametitle{splitting screen}
\begin{columns}
\begin{column}{5cm}
\begin{itemize}
\item Beamer
\item Beamer Class
\item Beamer Class Latex
\end{itemize}
\end{column}
\begin{column}{5cm}
\begin{tabular}{|c|c|}
\hline
\textbf{Instructor} & \textbf{Title} \\
\hline
Sascha Frank &  \LaTeX \ Course 1 \\
\hline
Sascha Frank &  Course serial  \\
\hline
\end{tabular}
\end{column}
\end{columns}
\end{frame}

\subsection{Pictures}
\begin{frame}\frametitle{pictures in latex beamer class}
\begin{figure}
\caption{show an example picture}
\end{figure}
\end{frame}

\subsection{joining picture and lists}

\begin{frame}
\frametitle{pictures and lists in beamer class}
\begin{columns}
\begin{column}{5cm}
\begin{itemize}
\item<1-> subject 1
\item<3-> subject 2
\item<5-> subject 3
\end{itemize}
\vspace{3cm}
\end{column}
\begin{column}{5cm}
\begin{overprint}
\end{overprint}
\end{column}
\end{columns}
\end{frame}


\subsection{pictures which need more space}
\begin{frame}[plain]
\frametitle{plain, or a way to get more space}
\begin{figure}
\caption{show an example picture}
\end{figure}
\end{frame}



\end{document}